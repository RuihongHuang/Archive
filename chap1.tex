\chapter{Introduction}
\label{Chapter:Introduction}

%big story
%general introduction to the task event extraction
Events are one important type of information throughout text. 
Accurately extracting significant events from large volumes of text 
informs the government, companies and the public regarding possible
 changing circumstances caused or implied by events. 


Event extraction is an information extraction (IE)
task that involves identifying 
%noun phrases%
entities and objects (mainly noun phrases) that represent 
%different key roles for
important roles in
events 
%in a particular domain.
of a particular type. The extracted noun phrases are called role fillers of events and 
they are the participants of events, objects that are involved in events, or properties 
associated with an event. For example, event extraction
systems for the terrorism domain identify
the perpetrators, victims, and targets of terrorism
events while systems for the management succession domain identify
the people and companies involved in corporate
management changes. 
%meaning
%Event extraction helps to detect and track events of significance and 
%informs the government, companies and the public about changes caused by events. 

However, extracting event information completely and accurately is challenging 
mainly due to the high complexity of discourse phenomena.
While this task has been studied over the last decades,  
%However, the current event extraction systems fail to extract many role fillers 
%and many noun phrases extracted are not correct role fillers in events.
%classifier/ pattern
%However, both the accuracy and the coverage of current event extraction systems are limited partly because 
%the design of most current event extraction systems is based upon two assumptions that are not always true.
 the performance of current event extraction systems is limited 
because they mainly consider local context 
%(mostly from the same sentence as the extracted noun phrase) 
(mostly isolated sentences) and ignore the influences of wider contexts 
from the discourse. 
% and separate event expressions
%when making extraction decisions.
My research aims to improve both coverage and accuracy of event extraction performance 
by exploring discourse-guided models. By incorporating discourse information 
beyond an individual sentence, the discourse guided models will seek out 
event information that tends to be overlooked by current event extraction systems 
and filter out extractions that seem to be valid when viewed locally.  
%incorporating discourse information beyond an individual sentence 
%and detecting multiple facets essential to events.

Finding documents that describe a specific type of event is also challenging 
because of the wide variety and ambiguity of event expressions. 
My research also aims to accurately identify event relevant documents 
 by proposing multi-faceted event recognition. Event facets represent 
event defining characteristics. Multi-faceted event recognition uses 
event facets, in addition to event expressions, to effectively resolve 
the complexity of event descriptions.

%\section{Overview of the Research}
\section{Discourse Guided Event Extraction Architectures}
%the whole document
% and leveraging the dependencies among event role fillers.
%The design of most current event extraction systems is based upon two assumptions.

%The second assumption underlying most traditional event extraction systems is that 
%role filler extractions are always accompanied by context that explicitly refer to a relevant event and 
%local context is sufficient to make an event extraction decision.
%However, many role fillers occur in contexts that do not explicitly mention the event, and those
%fillers are often overlooked. For example, the perpetrator of a murder may be mentioned in the context 
%of an arrest, an eyewitness report, or speculation about possible suspects. 
%Victims may be named in sentences that discuss the aftermath of the event,
% such as the identification of bodies, transportation of the injured
% to a hospital, or conclusions drawn from an investigation. 
%These contexts are not specific to a particular type of event 
%but tend to occur together with the events in interest, e.g., terrorism events.
%Role fillers can also be overlooked because a sentence
%does not appear to be relevant when viewed in isolation.
Most current event extraction systems 
%use either a
%learning-based classifier to label words as role
%fillers, or lexico-syntactic patterns to extract role
%fillers from pattern contexts.
%Both approaches, however, 
%generally tackle event recognition and role
%filler extraction at the same time and 
heavily rely on local contexts
 and individual event expressions
when making extraction decisions. 
%In other words, most event extraction systems 
They primarily recognize
contexts that explicitly refer to a relevant event and extract the noun phrases 
in those contexts as role fillers. 
For example, a system that extracts information about
murders will recognize expressions associated with
murder (e.g., ``killed'', ``assassinated'', or ``shot to
death'') and extract role fillers from the surrounding
context. 
%Relying on local contexts only 
However, lacking the view of wider context limits the performance of traditional event extraction systems in 
%the following 
two aspects. 

%Lacking the view of global contexts limits the performance of traditional event extraction systems. 
%The limitations are shown as both low recall and low precision of current systems.
First, the coverage of event extraction systems is limited because many role fillers occur in contexts
that do not explicitly mention the event, and those
fillers are often overlooked by current event extraction systems. For example, the perpetrator
of a murder may be mentioned in the context
of an arrest, an eyewitness report, or speculation
about possible suspects. Victims may be named
in sentences that discuss the aftermath of the event,
such as the identification of bodies, transportation
of the injured to a hospital, or conclusions drawn
from an investigation. I will refer to these types of
sentences as ``secondary contexts'' because they are
generally not part of the main event description (``primary contexts'').
Role fillers in secondary contexts are generally overlooked by current event extraction systems. 

However, extracting information from these secondary contexts indiscriminately
can be risky because secondary contexts occur with irrelevant events too.
For example, an arrest can follow a theft instead of a terrorism event.
This is why most current event extraction systems generally ignore the extractions in secondary contexts. 
Even within the main event description, a sentence may not 
appear to be relevant when viewed in isolation. 
For example, ``He used a gun''. Is the ``gun'' a weapon used in a terrorism event? Depending on the surrounding story context, 
such a sentence can be seen in the description of a terrorism event, a military operation event, or a common crime; accordingly, 
``He'' may refer to a terrorist, a soldier or a burglar.
However, if we know that the larger context is discussing a relevant event, 
%we are more certain that the secondary contexts are event relevant.  
then we will be able to extract relevant event information from these contexts and improve the coverage of event extraction systems.
%details to go to chap3
%In addition, we observed that documents of the narrative genre are abundant with secondary contexts.
%Therefore, we adopt a two-pronged strategy for
%event extraction that handles event narrative documents
%differently from other documents. We define
%an event narrative as an article whose main purpose
%is to report the details of an event. 
%We search for secondary contexts only in event narratives
%to aggressively extract role fillers in these stories.
%Following this key idea, our new event extraction system, called TIER, is able to find role fillers used to be missed by the previous systems and 
%improves the overall event extraction performance. 

Second, with access to wider context, the accuracy of current event extraction systems can be improved too.
%The reason is that depending on the larger contexts, 
%even the contexts which explicitly mention the relevant event may not be truly about the event due to ambiguity and metaphor. 
%Constrained by the limited view of contexts, 
Current event extraction systems will extract information 
if the local context 
%explicitly refer to a relevant event.
contains seemingly relevant event keywords or phrases. 
% event though they are not truly about the event due to ambiguity and metaphor.
However, depending on the larger context, they may not be 
%truly about the 
referring to a relevant event due to ambiguity and metaphor. 
For example, 
%people can be injured or killed in many ways (e.g.,
%vehicle accidents, military engagements, natural disasters,
%and civilian crime). Someone who is injured or killed should
%be characterized as a victimof terrorismonly if the discourse
%is describing a terrorist event. 
``Obama was attacked'' may lead to
Obama being extracted as the victim of a physical attack,
even if the preceding sentences describe a presidential debate
and the verb ``attacked'' is being used metaphorically.

Both of these problems tell us that it is necessary to 
develop better performing event extraction systems by modeling the influences of discourse during event extraction. 
In this dissertation, I will describe two discourse-oriented event extraction architectures that 
incorporate discourse information into event extraction to improve both extraction coverage and accuracy.
In the following two subsections, I will briefly describe the design of these two models.
%I have developed two discourse-oriented event extraction architectures to 
%incorporate discourse information beyond an individual sentence during event extraction.  
%The first one, called {\it TIER}, aims to improve the coverage of current event extraction systems by 
%tackling with {\it secondary contexts} in event descriptions. 
\subsection{TIER: a Multilayered Event Extraction Architecture}
The first one, called {\it TIER}, 
is a multilayered event extraction architecture that performs 
document level, sentence level and noun phrase level text analysis to progressively
``zoom in'' on relevant event information. 
{\it TIER} represents a two-pronged strategy for event extraction that
handles {\it event narrative} documents differently from other
documents. I define an event
narrative as an article whose main purpose is to report the details
of an event.
In contrast, I will refer to the documents that mention a relevant
event somewhere briefly, as fleeting references.
%  (e.g., a news article that reports on the bombing of a
% building or the assassination of a politician).
%  We train a classifier
% to identify event narrative documents that are relevant to a
% particular domain (e.g., terrorism).
%I apply the {\it role-specific sentence classifiers} only to event
%narratives to aggressively search for
%role fillers in these stories.
I search for role fillers only in secondary contexts that occur in event narratives. 
%However, other types of documents can mention relevant events too.
%The MUC-4 
%corpus, for example, includes interviews, speeches, and terrorist
%propaganda that contain information about terrorist events. 
%I will refer to these documents as {\it
%fleeting reference} texts because they mention a relevant event
%somewhere in the document, albeit briefly. To ensure that relevant
%information is extracted from \underline{all} documents, I also 
%apply a conservative extraction process to every document to extract
%facts from explicit event sentences.  
\begin{figure}[b]
 \centering
% \includegraphics[height = 2in]{figures/all_new_100_3.eps}
% \includegraphics[width = 2in]{figures/IR_curve_4.eps}
 \includegraphics[width = 6in]{figures/TIER_arch_full_3.eps}
 \caption{TIER: A Multi-Layered Architecture for Event Extraction}
\label{TIER_flow_chart}
\end{figure} 


The main idea of {\it TIER} 
%behind my approach 
is to analyze documents at multiple
levels of granularity in order to identify role fillers that occur in
different types of contexts. My event extraction model (as shown in Figure \ref{TIER_flow_chart}) progressively
``zooms in'' on relevant information by first identifying the document
type, then identifying sentences that are likely to contain relevant
information, and finally analyzing individual noun phrases to identify
role fillers. At the top level, I train a 
%text genre
document classifier to identify event
narratives. At the middle level, I create two types of
sentence classifiers.  {\it Event sentence classifiers} identify
sentences that mention a relevant event, 
%are associated with relevant events, 
and {\it
  role-specific sentence classifiers} identify sentences that contain
possible role fillers irrespective of whether an event is
mentioned. At the lowest level, I use {\it role filler extractors} to
label individual noun phrases as role fillers. 
As documents
pass through the pipeline, they are analyzed at different
levels of granularity. All documents pass
through the event sentence classifier, and event sentences
are given to the role filler extractors. Documents
identified as event narratives additionally pass
through role-specific sentence classifiers, and the
role-specific sentences are also given to the role filler
extractors.
The key advantage of this architecture is that it allows
us to search for information using two different principles: (1) we
look for contexts that directly refer to the event, as per most
traditional event extraction systems, and (2) we look for secondary
contexts that are often associated with a specific type of role
filler in event narratives. Identifying these {\it role-specific contexts} can root out
important facts would have been otherwise missed. 


%The second event extraction architecture, called {\it LINKER}, focuses on improving the accuracy of event extraction while 
%maintaining a reletively high extraction coverage. 
\subsection{LINKER: a Bottom-up Event Extraction Architecture}
\begin{figure}[htbp]
 \centering
% \includegraphics[height = 2in]{figures/all_new_100_3.eps}
% \includegraphics[width = 2in]{figures/IR_curve_4.eps}
 \includegraphics[width = 3.5in]{figures/bottom_up_v4.eps}
 \caption{LINKER: A Bottom-up Architecture for Event Extraction}
\label{LINKER_flow_chart}
\end{figure} 
The second model, called {\it LINKER} (as illustrated in Figure \ref{LINKER_flow_chart}), is a 
%more principled 
%more 
unified discourse-oriented event extraction architecture. 
%compared to {\it TIER}. 
%In {\it LINKER}, 
In addition to a set of local role filler extractors as 
%the ones 
normally seen in event extraction systems,  {\it LINKER} uses a single 
sequentially structured sentence classifier to explicitly model the contextual influences across sentences 
and identify event-related story contexts.
% utilizing a series of textual cohesion properties across sentences.
The structured learning algorithm, conditional random fields (CRFs), explicitly models whether the previous
sentence is an event context, which captures discourse continuity across sentences.
%The novelty of LINKER is that it includes a structured sentence classifier to model
Furthermore, the structured sentence classifier can model 
% the textual cohesion properties across sentences.
a variety of discourse information as textual cohesion properties across sentences.
% are modeled in the structured sentence classifier. 
% based on both intra-sentential and inter-sentential information. 
%Compared to TIER, 
%Discourse information is explored thoroughly in the structured sentence classifier.  
%The sentence classifier uses a structured learning algorithm, conditional
%random fields (CRFs).
Features are designed to capture 
%a variety of discourse properties that include 
lexical word
associations, e.g., it is common to see ``bombed'' in one sentence and ``killed'' in the next sentence 
because bombing event descriptions are often followed by casualty reports.
Features are also designed to capture 
discourse relations across sentences, e.g., if two sentences are in a causal relation, 
then probably both are event relevant sentences or neither of them is.
In addition,
its bottom-up design allows distributional properties of the candidate role fillers within
and across sentences to be modeled as features. Intuitively, the presence of multiple role
fillers within a sentence or in the preceding sentence is a
strong indication that a relevant event is being discussed.

In {\it LINKER}, the sentence classifier sequentially reads a
story and determines which sentences contain event
information based on both the local and preceding contexts. 
Then, the structured sentence classifier and the set of local role filler extractors are combined by extracting only the candidate role
fillers that occur in sentences that represent event contexts, as determined by the sentence classifier. 
%I have done studies showing that there are two main types of event relevant documents, event narratives and fleeting references, and 
%the insufficient event contexts  are commonly seen in one type of event relevant documents, the event narratives. 
%By my definitions, event narratives are the articles whose main purpose is to report the details of an event while 
%fleeting references are the documents that mention a relevant event somewhere in the document, albeit briefly. 
%To accurately extract these role fillers, the event extraction systems should be able to associate these contexts with the event in interest.
%Therefore, I propose two discourse informed event extraction architectures that 
%can go beyond individaul sentence analysis and model the influences of discourse during event extraction.

\section{Multi-faceted Event Recognition}
%full/deep event analysis is expensive, minor point, accuracy: bigger point
Before giving documents to sophisticated event extraction systems, 
we want to ask if the documents actually contain any relevant events. 
Therefore, I also study event recognition that 
aims to identify documents describing a specific
type of event.  
%While event extraction studies aim to produce full representations of events, 
%event recognition aims to identify documents that describe a specific
%type of event.  
Accurate event recognition will improve event extraction accuracy 
because any extractions from documents that do not contain a relevant event 
will be false. 
%and hurt event extraction accuracy. 
%Furthermore, event extraction 
Furthermore, event recognition is essential to many other event oriented applications.
%, such as tracking events.
For example,   
with accurate event recognition, we can detect the first occurrences and
the following mentions of a particular type of 
%events, 
event, thus we can track the dynamics of
%certain type of events.
events over time.

Event recognition is a highly challenging task due to the high
complexity and variety of event descriptions.
%Another big flaw of most current event extraction systems is that 
%they mainly detect events by searching for 
%individual 
%event expressions alone.
It is tempting to assume that event keywords
are sufficient to identify documents that discuss instances
of an event. But event words are rarely reliable
on their own. 
For example, consider the challenge
of finding documents about civil unrest. The words {\it ``strike''},
{\it ``rally''}, and {\it ``riot''} refer to common types of civil unrest, but
they frequently refer to other things as well.  A  strike can refer
to a military event or a sporting event (e.g., {\it ``air
strike''}, {\it ``bowling strike''}), a rally can be a race or a
spirited exchange (e.g.,{\it ``car rally''}, {\it ``tennis rally''}), and a riot
can refer to something funny (e.g., {\it ``she's a riot''}).
% For natural disaster events, words like ``hurricane'' and ``tsunami''
% seem like relatively unambiguous event keywords. But both terms can
% also refer to sports teams, Hurricane is a city in Utah, and 
% Tsunami Research Center, tsunami of strangers, Hurricane Relief Fund
Event keywords also appear in general discussions that do not
mention a specific event (e.g., {\it ``37 states
prohibit teacher strikes''} or {\it ``The fine for inciting a riot is \$1,000''}).  
%Due to the ambiguity of single event keywords, the accuracy of event recognition is much degraded. 
%Second, event keywords can be mentioned in general discussions and
%used metophorically.%true to both keywords and phrases
Furthermore, many relevant documents are not easy to recognize
because events can be described with complex expressions that do not include 
event keywords. %recall misses  
For example, 
{\it ``took to the streets''}, {\it ``walked off their jobs''} and {\it ``stormed
 parliament''} often describe civil unrest. 
%This is problematic because many times, event expressions alone will not 
%unambiguously indicate an event of a particular type. 
%examples for three facets

I propose multi-faceted event recognition to accurately recognize 
event descriptions in text by identifying event
expressions as well as event facets, which are defining characteristics of the event.
%Event facets, which are defining characteristics of the event, 
%I call them ``event facets'', 
Event facets are essential to distinguish one type of event from another. 
%I define ``event facets'' as defining characteristics of the event. 
%Important event facets include agents, effects and purpose of events.
For example, given the event expression ``hit the village'', 
depending on the agents, it might refer to a natural disaster event if the agent is ``The flooding'', or 
it might be describing an air strike if the agent is ``The military bombs''. 
Given the event expression ``attacked'', 
depending on ``who'' were ``attacked'' as the patient, 
it can be associated with a terrorism event (``civilians'') 
or a general military operation (``soldiers'').
%if the extraction ``PEOPLE'' refer to ``civilians'', then it probably is associated with a terrorism events. 
%but if ``PEOPLE'' refer to ``soldiers'', then probably it is involved in a general military operation. 
%Another example is that given the event expression ``sat in a circle'', 
%if it is followed with an infinitive attachment ``to have the Friday beer'' to show the purpose, 
%then the text might be talking about a party, 
%however, if the attachment changes to ``to demand a salary raise'', then it might be reporting a civil unrest event.  
Furthermore, event facets are so powerful that frequently, 
events can be recognized 
by only seeing multiple types of event facet information, 
%with no 
without any event expression detected. 
For example, to identify documents describing civil unrest events, 
we feel confident to claim that a document is relevant if we pinpoint both the agent term ``coal miners'' 
and the purpose phrase ``press for higher wages'', event without detecting any event keyword such as 
``rally'' and ``strike''.
%Furthermore, before diving into the documents seeking for relevant event information, 
%better event extraction performance is achievable by carrying out more precise event document recognition. 
%In event extraction research, some data sets only include documents that describe
%relevant events (e.g., well-known data sets for the domains of
%corporate acquisitions~\cite{freitag-acl98,freitag-aaai00,finn04}, job
%postings \cite{califf03,freitag-aaai00}, and seminar announcements
%\cite{freitag-acl98,ciravegna01,chieu02,finn04,gu06}.  But
%many IE data sets present a more realistic task where the IE system
%must determine whether a relevant event is present in the document,
%and if so, extract its role fillers. Most of the
%Message Understanding Conference data sets represent this type of
%event extraction task, containing (roughly) a 50/50 mix of relevant and
%irrelevant documents (e.g., MUC-3, MUC-4, MUC-6, and MUC-7 \cite{hirschman98}).
%In the latter and more realistic setting, 
%%The first assumption is that \underline{all} the documents to be processed by event extraction systems contain event descriptions. 
%%However, 
%the documents fed into event extraction systems are generally acquired automatically using information retrieval (IR) techniques and 
%due to the event keyword ambiguities, many retrieved documents that contain keywords of one type of event in interest
%can be about completly different types of event. 
%For example, while the words ``strike'', ``rally'', and ``riot'' refer to common
%types of civil unrest, they frequently refer to
%other things as well. A strike can refer
%to a military event or a sporting event (e.g., {\it ``air
%strike''}, {\it ``bowling strike''}), a rally can refer to a race or a
%spirited exchange (e.g.,{\it ``car rally''}, {\it ``tennis rally''}), and a riot
%can refer to someone or something that is funny (e.g., {\it ``she's a riot''}).
%%many event irrelevant documents tend to be retrieved because they contain event keywords and are seemingly event relevant. 
%The retrieved event irrelevant documents will induce false extractions in event extraction systems.

%Therefore, I propose an multi-faceted event recognition approach that is 
%aware of event essential characteristics and able to detect events of a particular type accurately. 
\begin{figure}[htbp]
 \centering
% \includegraphics[height = 2in]{figures/all_new_100_3.eps}
% \includegraphics[width = 2in]{figures/IR_curve_4.eps}
 %\includegraphics[width = 5in]{figures/flow-chart_4.eps}
  \includegraphics[width = 5in]{figures/faceted/flow-chart_cu.eps}
 \caption{Bootstrapped Learning of Event Dictionaries, illustrated with Agents and Purpose as Event Facets.}
\label{multi_faceted_flow_chart}
\end{figure} 


%I also propose a 
The third component of my research is
%co-training 
bootstrapping framework to automatically learn event expressions as well as essential facets of events. 
%alternatively.
The learning algorithm relies on limited supervision, specifically, a handful of event keywords 
that are used to create a pseudo domain-specific text collection 
from a broad-coverage corpus, and several seed terms for each facet to be learned. 
%Suppose we focus on two event facets, agent and purpose.
%Experiments have been done on civil unrest events and the two event facets identified there are agent and purpose. 
% from a large text collection using a small amount of seeding. 
The learning algorithm exploits the observation that event expressions and event facet information 
%agents, and purpose phrases 
often appear together in sentences that introduce an event. 
Furthermore, seeing more than one piece of event information in a sentence tends to 
validate that the sentence is an event 
%introductory 
sentence and imply that 
additional event information may also be found in the same sentence.  
%The learning process uses unannotated texts, a few event
%keywords, and seed terms for each event facet associated with the event. 
%The event keywords will be used to create a pseudo domain-specific collection of articles  
%that are selected from a broad coverage corpus and contain at least one event keyword. 
Therefore, in the first step, I identify probable event sentences that contain multiple types of 
event facet information and extract event expressions 
% from syntactic positions
based on dependency relations with
event facet phrases. The harvested event expressions are added
to an event phrase dictionary. In the second step, 
% the event phrase dictionary is used to learn agent and purpose
% phrases. 
new phrases of an event facet are extracted
from sentences containing an event phrase and phrases of the other event facets. 
The newly harvested event facet phrases are 
added to event facet dictionaries. 
The bootstrapping algorithm ricochets back and forth, alternately 
learning new event phrases 
%in the first step 
and learning new 
%agent/purpose 
event facet phrases
%in the second step
, in
an iterative process. 
%Take an example type of events with agents and purpose as its event facets, 
For example, civil unrest events are generally initiated by certain population groups, 
e.g., ``emplyees'', and with certain purpose, e.g., ``demanding for better working conditions''. 
Therefore, I identify agents and purposes as two facets of civil unrest events.  
To learn event expressions and event facet phrases, 
Figure \ref{multi_faceted_flow_chart} illustrates how the bootstrapping algorithm works.

%The learned event expressions and event facet information can be used to improve event extraction performance in different levels of text processing. 
%First, they can be used to detect documents containing events of a particular type.
%Second, 
%the documents that describe events can be further categorized based on 
%the densities and positions that event expressions and event facet information are observed in text, 
%and accordingly, different event extraction strategies may be applied depending on the document categories.
%Third, event expressions and event facet information can be used to extract part of the event information because
%some event role fillers are contained within them.

%Most of the Message Understanding Conference data sets represent this type of
%event extraction task, containing (roughly) a 50/50 mix of relevant and
%irrelevant documents (e.g., MUC-3, MUC-4, MUC-6, and MUC-7 \cite{hirschman98})


\section{Claims and Contributions}

  % The thesis statement for this research
The primary contributions of this research are as follows:
%  \subsection{Thesis Statement}
%  \label{sec:thesis-statement}
  
%  {\em{Accuracy and coverage of event extraction can be improved by multi-faceted event recognition and discourse informed event context finding.
%incorporating discourse information across sentences 
%%when making extraction decisions and 
%%making multiple extraction decisions simultaneously by
%modeling correlations between role fillers
%% associated with the same event or related events
%}
%}
  \begin{enumerate}
    \item[1] {\em{Both event extraction coverage and accuracy can be improved by 
    incorporating discourse information across sentences 
    %in event extraction models.
    to recognize event contexts before applying local extraction models.}}

Event story telling generally spans over a text discourse. Accordingly, automatic event extraction systems should be able to 
%go beyond processing individual sentences and
 model discourse phenomena to accurately locate pieces of event information in text. 
%In this research, I focused on addressing the major limitation of 
%most current event extraction systems that generally process one sentence each time,  
However, current event extraction systems generally process one sentence a time and make extraction decisions relying on event clues 
from a limited text span as within the sentence.
Due to lacking a global view of text contents, the current event extraction systems suffer from insufficient coverage and accuracy. 
%Consequently, event information that is not accompanied by clear event indicators in proximate contexts tend to be overlooked by the current event extraction systems. 
%Furthermore, the current systems produce wrong extractions that seem relevant when viewed locally. 
In this research, I focus on 
%addressing the issue of second contexts in event descriptions and 
improving extraction performance by incorporating discourse information across sentences 
%in discourse guided event extraction models. 
to recognize event contexts before applying local extraction models
First, I designed a multi-layered event extraction model, called {\it TIER}, 
to seek out event information that appear in secondary event contexts. 
The main idea of {\it TIER} is to zoom in on relevant event information, 
by using a document classifier and two types of sentence classifiers to analyze text 
at multiple granularities.
Later, I designed a 
%more principled 
unified discourse guided event extraction architecture, {\it LINKER}, 
that explicitly model textual cohesion properties across sentences
 to accurately find out event related contexts, using a single structured sentence classifier. 
Evaluation on two event domains shows that my discourse guided event extraction architectures have improved both 
event extraction coverage and accuracy.

    \item[2] {\em{Event defining characteristics (event facets), in addition to event expressions, can be used to accurately identify documents describing 
a particular type of event.}}

Finding documents that describe a specific type of event is a 
%highly 
challenging task due to the high complexity and variety
of event descriptions. Event keywords tend to be ambiguous and are not sufficient to identify documents that discuss 
event instances of a particular type.
I propose multi-faceted event recognition to accurately recognize event descriptions in
text by identifying event expressions as well as event facets, which are defining characteristics of the event. 
Event facets, such as agents, purpose and effects of events, are essential to distinguish one type of event from another.
I also propose a bootstrapping framework to automatically learn event expressions as
well as essential facets of events, requiring only unannotated text and minimal human supervision.
Evaluation on two event domains shows that multi-faceted event recognition can yield high accuracy.


  \end{enumerate}

  % Goals and objectives of the proposed research.
%  \subsection{Goals \& Objectives}
%  \label{sec:goals-and-objectives}
  
%  The goals and objectives of this research are the following:
%  \begin{enumerate}
%    \item[(a)] To design new event extraction models to incorporate discourse information across sentences.
%    %\item[(b)] To use multi-faceted event document recognition to achieve accurate event document identification.
%    \item[(c)] To learn and utilize event facet information to categorize event documents and extract event information. 

%    %\item[(c)] 
%  \end{enumerate}

  % Finally, the primary idea of this thesis -- needs a rewrite
%   There is a simplicity and elegance in having a single model that takes on both
%   of the responsibilities of IE. However, the hypothesis of this research is that
%   there are benefits to decoupling these into separate tasks. This research
%   investigates an approach for IE, which involves two passes over a document. In
%   the first pass, an automated system separates regions of text as being relevant
%   or irrelevant to the domain of interest. In the second pass, a number of local
%   clues associated with spans of text within these regions are used for the
%   extraction of some of these spans as event information. This very basic idea
%   has numerous ramifications, and motivates several possibilities for better IE,
%   which would not have been possible with a monolithic system. This proposal
%   discusses the design and implementation of this two-phase system in acute
%   detail, and additionally, enumerates various possibilities for IE performance
%   improvement enabled by this approach.


\section{Navigating this Dissertation}
The rest of this dissertation is summarized as follows. 
  \begin{itemize}
   \item Chapter \ref{Chapter:Background} describes existing work in event extraction and recognition studies.
This chapter explains the limitations of current event extraction and recognition systems 
and demonstrates how the research presented in this dissertation contributes in addressing these limitations. 

   \item Chapter \ref{Chapter:TIER} presents the details in designing
   %and implementing 
   {\it TIER}, 
the multilayered event extraction architecture, which can seek out event information 
from seconday event contexts. To motivate, I will first discuss 
secondary contexts, in contrast with primary event contexts. 
%that are distinguished from primary event contexts, are discussed. 
%In addition, 
Then, I will demonstrate two types of documents that mention relevant events,  
event narratives and fleeting references, 
%identified as two types of documents 
%that mention relevant events, are also discussed. 
with respect to how event information was conveyed in text.  
Then this chapter presents design 
%and implementation 
details of the four components 
that constitute the multilayered event extraction architecture. 
  

   \item Chapter \ref{Chapter:LINKER} presents the details in designing 
   %and implementing the more principled
   the unified 
discourse guided event extraction architecture, {\it LINKER}. This chapter illustrates 
the bottom up system design and discusses the main idea that 
uses a single structured event context 
recognizer to identify all the event related sentences in a document. 
After that, this chapter elaborates the linguistic discourse features that are used in the 
structured event context recognizer to 
capture textual cohesion properties aross sentences.


   \item Chapter \ref{Chapter:Multi-faceted} demonstrates multi-faceted event recognition approach. This chapter 
discusses the insufficiency of using event keywords for event recognition and 
illustrates how event defining characteristics (facets) can be helpful to recognize events of a particular type in text.
This chapter includes a thorough discussion of event facets in a variety of events. 
Then this chapter describes details of the bootstrapping framework that is effective in 
acquiring both event expression and event facet information from unannotated text. 
In the evaluation section, I also examines whether 
multi-faceted event recognition 
can be used to improve event extraction performance, 
especially with respect to extraction accuracy. 

%   \item Chapter \ref{Chapter:Stratified} examines whether the automatically acquired event expression and event facet information 
%for event recognition purpose  
%can benefit event extraction, via filtering and categorizing documents based on the number of relevant events 
%mentioned in the documents. This chapter presents empirical evaluations using existing event extraction 
%datasets on two event domains.

   \item Chapter \ref{Chapter:ConclusionsAndFutureDirections} discusses the conclusions that we can draw from the dissertation. 
Following the conclusions, this chapter 
%sheds light on 
suggests the future directions that can potentially lead to 
further progress in event oriented information extraction research.
 
  \end{itemize}

