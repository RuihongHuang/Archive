\chapter{Conclusions and Future Directions}
\label{Chapter:ConclusionsAndFutureDirections}
%future directions: using knowledge based inference to improve event extraction
%future directions: learn associated events to complement main event features acquired by multi-faceted
In this chapter, I will first summarize my event extraction and recognition research 
and contributions. Then, I will discuss several future directions that may lead to 
futher improved event extraction and recognition performance. 

\section{Research Summary and Contributions}

My research has been concentrated on improving 
event extraction performance by 
explicitly identifying 
%recognizing 
event contexts 
%within a document 
%accurately 
before 
extracting individual facts. 
%applying local extraction models. 
In this section, 
%I will summarize my research by 
I will first emphasize the problems and limitations of current 
event extraction systems, which have motivated my research, 
then I will briefly go through the approaches and algorithms that 
I introduced to improve event extraction performance 
and highlight my contributions. 

Most current event extraction systems 
heavily rely on local contexts
 and individual event expressions 
 to recognize event mentions 
when making extraction decisions. 
However, lacking the view of wider context limits both 
the extraction coverage and the accuracy of traditional event extraction systems. 
The coverage of event extraction systems is limited 
because many role fillers occur in contexts
that do not explicitly mention the event, 
e.g., the perpetrator
of a murder may be mentioned in the context
of an arrest, and those
fillers are often overlooked. 
%by current event extraction systems. 
The accuracy of current event extraction systems 
is limited too because 
%event extraction 
%will extract information 
even if the local context 
%explicitly refer to a relevant event.
contains seemingly relevant event keywords or phrases. 
% event though they are not truly about the event due to ambiguity and metaphor.
%However, 
depending on the larger context, they may not be 
%truly about the 
referring to a relevant event due to ambiguity and metaphor. 
For example, ``Obama was attacked'' may lead to
Obama being extracted as the victim of a physical attack,
even if the preceding sentences describe a presidential debate
and the verb ``attacked'' is being used metaphorically.
%For example, 
Therefore, by only considering local contexts, 
current event extraction systems can generate many false extractions.

To address 
%the above 
these limitations of current event extraction systems and  
improve event extraction performance, 
I proposed two new event extraction architectures that 
incorporate discourse information across sentences 
%in discourse guided event extraction models. 
to recognize event contexts before applying local extraction models. 
First, to seek event information out of secondary contexts and 
thus improve the coverage of event extraction performance, 
I created {\it TIER}, a multilayered event extraction architecture 
that performs document level, sentence level and noun phrase level 
text analysis to progressively 
``zoom in'' on relevant event information.
The challenge for extracting information from secondary event contexts 
is that 
%extracting information from these secondary contexts indiscriminately
%can be risky because 
secondary contexts occur with irrelevant events too.
For example, an arrest can follow a theft instead of a terrorism event.
Keeping this in mind, 
{\it TIER} represents a two-pronged strategy for event extraction that 
distinguish two types of documents that mention relevant events, 
{\it event narratives}  v.s {\it fleeting references}, 
and only 
%consider to 
extracts information from secondary contexts in 
{\it event narratives}. 
Event narratives are articles whose main purpose 
is to report the details of an event while 
fleeting references are the documents that only briefly mention a relevant
event, but do not elaborate on the event details. 

To make event extraction systems more precise,  
I also proposed a 
%unified 
discourse-guided event extraction model, 
called {\it LINKER}.
In addition to 
a set of local role filler extractors as 
%the ones 
normally seen in event extraction systems, 
{\it LINKER}  
includes a structured sentence classifier 
that sequentially reads a
story and determines which sentences contain event
information based on both the local and preceding contexts.
Then, the structured sentence classifier and 
the set of local role filler extractors are combined 
by extracting only the candidate role
fillers that occur in sentences that 
represent event contexts, as determined by the sentence classifier.
%In summary
Specifically, 
the structured learning algorithm, 
conditional random fields (CRFs), 
explicitly models whether the previous
sentence is an event context, 
which captures discourse continuity across sentences.
Furthermore, the structured sentence classifier 
uses well designed features to 
capture textual cohesion properties across sentences, 
including lexical word
associations, discourse relations across sentences,  and 
distributional properties of the candidate role fillers within
and across sentences. 

Another issue of current event extraction systems is 
that they do not 
%distinguish if the 
attempt to determine whether a document 
%to be 
%extracted information from 
contains any relevant information 
%or not. 
before extracting facts based only on local context. 
Processing documents that do not mention a relevant event 
%for extraction purpose 
is 
%meaningless and 
a waste of computing resources. 
Furthermore, 
accurate event recognition will improve event extraction accuracy 
because any extractions from irrelevant documents 
%that do not contain a relevant event 
will be false. 
%and hurt event extraction accuracy. 
However, identifying documents that mention an event of a particular type 
is a highly challenging task due to the high
complexity and variety of event descriptions. 
Event keywords are rarely reliable
on their own. 
For example, consider the challenge
of finding documents about civil unrest. The words {\it ``strike''} and
{\it ``rally''} refer to common types of civil unrest, but
they frequently refer to other things as well.  A  strike can refer
to a military event or a sporting event (e.g., {\it ``air
strike''}, {\it ``bowling strike''}), and a rally can be a race or a
spirited exchange (e.g.,{\it ``car rally''}, {\it ``tennis rally''}). 
I proposed multi-faceted event recognition to accurately recognize 
event descriptions in text by identifying event
expressions as well as event facets, which are defining characteristics of the event.
%Event facets, which are defining characteristics of the event, 
%I call them ``event facets'', 
Event facets, such as agents, purpose and effects of events, 
are essential to distinguish one type of event from another. 
%I define ``event facets'' as defining characteristics of the event. 
%Important event facets include agents, effects and purpose of events.
For example, given the event expression ``hit the village'', 
depending on the agents, it might refer to a natural disaster event if the agent is ``The flooding'', or 
it might be describing an air strike if the agent is ``The military bombs''. 

I also proposed a bootstrapping framework to 
automatically learn event expressions as well as essential facets of events, 
which 
%only 
relies on limited human supervision. 
The learning algorithm exploits the observation that event expressions and event facet information 
%agents, and purpose phrases 
often appear together in sentences that introduce an event. 
Furthermore, seeing more than one piece of event information in a sentence tends to 
validate that the sentence is an event 
%introductory 
sentence and 
%imply 
suggests that 
additional event information may also be 
%found 
in the same sentence. 
Therefore, the bootstrapping algorithm ricochets back and forth, alternately 
learning new event phrases 
%in the first step 
and learning new 
%agent/purpose 
event facet phrases, 
%in the second step
in an iterative process.  

After reflection on my research, I have focused on improving 
coverage and accuracy of event extraction systems by recognizing event contexts 
before applying extraction models. I investigated 
event context identification by both recognizing documents that mention 
a relevant event and finding event regions within a document. 
%event relevant documents that describe a and event re the three specific contributions 
Specifically, my main contributions are as follows:

  \begin{enumerate}
   \item[1] {\em{I distinguish secondary event contexts from primary contexts and propose 
   a multi-layered event extraction architecture that can seek out event information from 
   different types of event contexts.}}
   
  Many event role fillers occur in secondary contexts that do not explicitly mention the event  
%I refer to them as ``secondary contexts'' because they are
and are generally not part of the main event description (``primary contexts''). 
Event role fillers 
in secondary contexts 
are generally overlooked by current event extraction systems. 
To seek event information out of secondary contexts, 
I introduced {\it TIER}, a multilayered event extraction architecture 
that performs document level, sentence level and noun phrase level 
text analysis to progressively ``zoom in'' on relevant event information. 

   \item[2] {\em{I proposed a 
   %unified 
   discourse-guided event extraction architecture that 
   uses a single structured event sentence classifier to capture various textual 
   cohesion properties and identify 
   %all 
   event contexts in a document.}}
%Later, I realized that both secondary contexts and primary contexts 
%should be validated using larger scope discourse information. 

The discourse-guided event extraction architecture is called {\it LINKER}. 
In addition to a set of local role filler extractors, 
%as 
%the ones 
%normally seen in event extraction systems,  
{\it LINKER} uses a single 
sequentially structured sentence classifier to explicitly 
model the contextual influences across sentences 
and identify event-related story contexts. 
In the structured sentence classifier, a variety of discourse information 
is modeled as textual cohesion properties across sentences, including 
lexical cohesion features, discourse relations and domain-specific 
candidate role filler distributional features. 

   \item[3] {\em{
   %Event defining characteristics (event facets), 
   %in addition to event expressions, can be used to accurately identify documents describing 
%a particular type of event.
I proposed multi-faceted event recognition, which 
uses event defining characteristics, in addition to event expressions, to identify documents 
describing a particular type of event.}}

Finding documents that describe a particular type of event is a 
%highly 
challenging task due to the high complexity and variety
of event descriptions. Event keywords tend to be ambiguous 
and are not sufficient to identify documents that discuss 
events of a specific type.
I observed that event defining characteristics, I call them event facets, 
such as agents, purpose and effects of events, 
are essential to distinguish one type of event from another.
Therefore, I proposed multi-faceted event recognition to 
accurately recognize event descriptions in
text by identifying event expressions as well as 
event facets. 
I also proposed a bootstrapping framework to 
automatically learn event expressions as
well as essential facets of events from free texts, which 
only requires 
%unannotated text and 
minimal human supervision.
%Evaluation on two event domains shows that multi-faceted event recognition can yield high accuracy.
  \end{enumerate}

\section{Future Directions}

Evaluation of my research using two distinct event domains 
has shown that the 
discourse-guided event extraction architectures can  
improve both coverage and accuracy of event extraction performance 
and the multi-faceted event recognition can 
%effectively 
effectively identify event relevant documents and further 
improve event extraction accuracy. 
However, we can see that overall, 
the extraction performance is far from 
%being 
perfect. 
Furthermore, 
most event extraction systems heavily rely on 
human annotated data to train event extraction systems for each 
%event domain 
type of event. 
It is important to reduce the dependence on human supervision 
%dependent 
to be able to quickly configure domain-specific event extraction systems.
%On one hand, this indicates that event extraction is a 
%challenging natural language processing task. 
%On the other hand, a significant amount of future work is 
%expected to further study this task. 
%and improve event extraction performance. 
In the following subsections, I will discuss 
several thoughts I have when I meditate on the research presented  
in this dissertation and 
%they 
present some ideas that 
may lead to better event extraction performance 
or reduce human supervision that is required to train event extraction systems.  

\subsection{Incorporating Domain Knowledge to Tackle Inferences}
%what types of knowledge, how to incorporate, how to acquire
Most current event extraction systems are  
automatically trained using human labeled data as supervision 
and use limited 
%amount 
amounts of external 
domain knowledge.
Therefore, systems trained this way heavily rely on surface textual clues, 
word forms or shallow semantics of words, to 
determine if an event is being described and what type of event information 
is being conveyed. 
For instance, if a person has been annotated as a 
victim of terrorism events several times and each time, the textual contexts 
that follow the extractions are the same, e.g., ``was killed'', 
then the automatically trained event extraction systems 
can learn the pattern that whenever a person was followed by 
``was killed'', the person 
%mention 
should be extracted as a victim.  
However, this 
%can only partially imitate how 
%we process information. 
is clearly not sufficient to 
imitate how humans process information.

Generally, we go through various kinds of inferences at multiple levels 
when we read texts and digest information described in texts.
For example, 
%if 
suppose a document describes a bombing event 
in a shopping mall and later discusses a police investigation on a 
man in a hat that appeared in the monitored video of the shopping mall 
before the bombing event happened.  
%then we 
We can easily infer that the investigated man has 
been suspected to have carried out the bombing and 
should be extracted as the potential perpetrator of the bombing event. 
In my work, I distinguish secondary contexts from 
primary contexts and proposed the multi-layered event extraction system, 
{\it TIER}, to seek out information from the contexts that do not 
%clearly state 
explicitly refer to 
the event. In this case, the investigation context 
can be viewed as a secondary context. 
However, without plugging in explicit inference driven components, 
the automatically trained event extraction systems are good at 
recognizing {\it textual patterns} that have repeated themselves 
many times, but tend to miss the less frequent ones. 
Unfortunately, due to the diversity of event descriptions, 
a large proportion of textual clues only  
occur occasionally in the annotated data.  
%Therefore, if 

However, with the aid of a rich set of domain knowledge, 
we can design extraction components and mechanisms 
that carry out inferences similar to the human. 
Therefore, we possibly obtain opportunities to 
automatically make intelligent extraction decisions 
%extract event information from 
with many less common
contexts. 
%and make intelligent extraction decisions. 
In the next section, I will present my thoughts 
on the specific types of domain knowledge that may 
be helpful to event extraction. 




\subsection{Acquiring Domain Knowledge from Unlabeled Texts}
\label{knowledge-types}

As discussed in the previous subsection, 
inference driven event extraction might enable better extraction performance.
%While we are not so clear about how we can effectively incorporate 
%domain knowledge to make intelligent extraction decisions, 
%we should start from acquiring domain knowledge. 
Ideally, we should design algorithms that can acquire 
domain knowledge 
from a large 
%scale 
volume of unlabeled and easily accessible plain texts,  
which can reveal {\it textual patterns} that may have appeared 
only one time in a limited size of annotated data set.


The primary question is what type of domain knowledge 
 we should aim to acquire. 
%I have identified several categories of  
The first category of domain knowledge I have identified 
is events that are somehow related to the target event.   
%Real world events never happen or progress independently. 
%Instead, often, 
Events are often causally
linked, temporally connected, or mutually influenced. 
%This event nature 
This nature of event 
is also manifested in their textual descriptions. 
For instance, articles that mainly describe an event also 
discuss its related events that may have caused 
the target event or happened as the consequences of the target event.
Therefore, going beyond examining the descriptive nature 
of individual events, studying the relations between different 
types of events will help us to 
%understand 
%how certain events happen and 
better detect the relevant event 
descriptions and locate event information. 
I am especially interested to answer the question of 
how one event occurs in the contexts of a variety of
the other events and what are the types of associations among events.

The second type of domain knowledge that 
may benefit event extraction performance is sub-events of the 
target event. 
I have observed that in event descriptions, 
event information is frequently mentioned in 
%the 
detailed 
elaborations 
%on 
of how the event happened. 
Therefore, knowledge about routine sub-events of a particular type of event 
should be helpful for event extraction systems to identify 
specific types of event information. 


 
\subsection{Building Event Ontologies}

Most current event extraction systems 
are trained for predefined types of events. 
Mainly, training event extraction systems to tackle 
certain type of events can effectively reduce 
the complexity of text analysis needed to 
detect and extract event information, however, 
it also implies that the extraction systems 
should be retrained whenever new types of 
events are targeted. 
%Therefore, the 
To resolve this dilemma, one possible solution is 
to build up an event ontology and associate 
the automatically acquired domain knowledge to 
its corresponding type of event in the event ontology. 

To achieve this goal, we have to answer questions about 
the structure of an event ontology. 
For example, what criteria should we use to categorize  
events and how many main event classes should be included 
in the event ontology. 
Possibly, we can use event facets to 
group events and events sharing the same set of 
essential event facets form a class of events. 
The other type of questions are about when we  
should build links between event classes and 
what the inter-event relations can be. 
Potentially, we can use the same set of 
inter-event relations as discussed in Subsection 
\ref{knowledge-types}, including 
causal, temporal relations and the relation that 
an event is a subevent of the other event. 

In short, a well structured event ontology and 
a rich collection of domain knowledge that 
is mapped to each type of event in the ontology 
are valuable to both further improve the performance 
of event extraction systems and reduce their dependence 
on human supervision. 
First, domain knowledge of a specific type of event and 
domain knowledge of its related events can be explored   
to develop inference driven event extraction systems.  
And these systems have potential to make intelligent extraction 
decisions and enhance extraction performance. 
Second, event domain knowledge and structured ontology 
represent generalizations and summaries of 
diverse forms of events, therefore, 
the access of such knowledge can 
make the system training less dependent on 
human supervision that is often realized by 
annotating specific event descriptions. 



