\chapter{CU Event Role Filler Annotation Guidelines}
\label{Appendix:EventRoleFillerAnnotationGuidelines}
This appendix lists the civil unrest event role filler annotation guidelines that were provided
to the annotators.

%\section{Civil Unrest Event Definition}

%Civil unrest (CU) is a broad term that is typically used by the media or
%law enforcement to describe a form of public disturbance caused by a
%group of people for a purpose. Civil unrest events include activities to protest
%against major socio-political problems, events of activism to support
%a cause (e.g., peace rallies or large-scale marches to support a
%prominent figure), and events to promote changes in government or
%business affairs (e.g., large gatherings to rally for higher wages).
%Types of civil unrest can include, but are not necessarily limited to:
%strikes, rallies, sit-ins and other forms of obstructions, riots,
%sabotage, and other forms of public disturbance motivated by a
%cause. It is intended to be a demonstration to the public, the
%government, or an institution (e.g., business or educational sectors),
%but can sometimes escalate into general chaos.\\

%{\bf Civil Unrest Events to be Considered}
%CU Events to be Considered:
%\begin{enumerate}
%\item[{\bf 1.}] According to the definition, CU events do not include war, ethnic fightings 
%or fightings involving armed parties only. 
%\item[{\bf 2.}] Descriptions of currently on-going or recently (within one year) 
%happened CU events should all be examined for annotations. Old CU events 
%that happened more than one year ago from the date when the article was 
%published should not be examined for annotations. 
%\item[{\bf 3.}] In addition to the CU events that actually happen, descriptions of 
%threatened and planned CU events that may happen should also be examined 
%for annotations even if there is uncertainty or they are mentioned conditionally. 
%However, CU event descriptions of threatened and planned CU events that 
%will definitely not happen should not be examined for annotations.
%\item[{\bf 4.}] Event summary information that is synthesized from two or more events 
%should NOT be annotated.
%\end{enumerate}

\section{Annotation Task Definition}

You will need to read a set of news articles and identify the CU event 
descriptions. Specifically, you need to find out phrases in text that 
describe CU events and put them into their corresponding slots. 
Each slot indicates one type of event information, such as locations, 
agents, causes and damages of CU events.\\

{\bf IMPORTANT}
\begin{enumerate}
\item[{\bf 1.}] More than one event can be described in an article, however, only one set of 
event role slots are to be filled out. Therefore, you should put all the phrases 
fulfilling a specific event role into the same slot EVEN THEY ARE FROM DIFFERENT EVENTS.
\item[{\bf 2.}] Events that are described in an article are often related. If the same entity 
or object is involved and plays the same role in multiple events, you should only 
annotate the mentions across all its mentions that are significantly different in 
lexical forms.
\end{enumerate}



\section{The Event Slots to be Annotated}
Eight event slots will be considered for annotations.\\

\begin{enumerate}

\item[{\bf (1)}] {\bf Event type (Closed Set)}

Instead of labeling strings in text like (1), choose
 event types from the following CU types:\\

STRIKE(S) -- consists of refusing to work.\\
MARCH(ES) -- consists of moving from one place to another place.\\
SIT-IN(S)/OCCUPATION(S) -- consists of taking up some space and thus disturbing the 
regular activities that require the space.\\
OTHER(S) -- other forms of protest(s) or demonstration(s), such as rally (rallies), 
riot(s), sabotage(s), etc. \\

If one CU event includes activities of type STRIKE (or MARCH, or OCCUPATION), 
select STRIKE (or MARCH, or OCCUPATION) as the event type, even the event 
described also includes activities of other types. If one event includes 
multiple types of activities, e.g. both STRIKE and OCCUPATION, select all the 
appropriate types. Select types for all CU events described in an article, e.g., 
select both STRIKE and MARCH if two CU events were described in an article, one 
event is of type STRIKE and another is of type MARCH. You should select OTHER 
if none of the CU events described in an article includes activities of any of 
the first three types. 

\item[{\bf (2)}] {\bf Agent/Population}

The population GROUPS who initiate, lead or join to strengthen 
the CU events. If there are multiples references to roughly the same population group, 
label all the ones that are in different lexical forms, including the general mention terms 
such as "the protesters".

\item[{\bf (3)}] {\bf Key Issue} 

The most essential issues that motivate the CU events. 
Sometimes, you will see multiple issues/causes that are likely to be motivating 
the CU events, but ONLY label the MOST direct/explicit one. Key issues can be 
natural resources, abstract concepts, actions, decisions and events that are 
demanded or protested against or others. 
For example, in
  \begin{quote}
 {\em The workers went on strike to press for higher wages.}
  \end{quote}
  you should  put ``higher wages'' as the key issue.
  
Issues can be described as base 
noun phrases and base verb phrases, if you think parts of the essential issues 
are included in pp attachments (if any) of the base noun/verb phrases, label 
the pp attachments too, furthermore, if you think a complete clause is needed 
to describe an issue, label the full clause. 
For instance, consider
  \begin{quote}
   {\em The demonstrators protest against Spain's proposal to legalize 
 gay marriage.}
  \end{quote}
you should put ``legalize gay marriage'' as the key issue without 
 including ``Spain\'s proposal to'' because ``Spain\'s proposal to'' only signify the 
 source (Spain) and the state (being proposed) of the key issue.
 
 Another example is as follows,
  \begin{quote}
   {\em The youth of Lebanon have occupied the front line of the "Cedar Revolution" 
 gripping the country since the February 14 assassination of popular former prime 
 minister Rafiq Hariri.}
  \end{quote}
you should label ``the February 14 assassination of popular 
 former prime minister Rafiq Hariri'' as the key issue. 
 
If there are multiple references (can be 
in any of the above three forms) to the essential issue (should be only one in content 
for one event), please label all the ones that are in different lexical forms.

  


\item[{\bf (4)}] {\bf Site\/Facility}

A human constructed facility where a CU event takes place. 
A site can be a plaza, a shopping mall, a mosque, a bridge, a hospital or an university.

\item[{\bf (5)}] {\bf Location}

NAMED geographical regions/areas where a CU event takes place. 
A location can be a city (e.g. ``Beijing''), a country (e.g. U.S.) or other named places 
(e.g. Antelope Island). Only label the location names themselves. You should only 
consider the locations that appear in the context of a CU event. You should not consider 
the locations that are embedded in organization names. 
For instance, in 
  \begin{quote}
   {\em 25,000 opposition supporters demonstrated in Lome, the capital of Togo.}
  \end{quote}
 You should label ``Lome'' and ``Togo'' as two locations and put them in two lines. 

\item[{\bf (6)}] {\bf HumanEffects (Injuries and Deaths)}

The casualties that are {\em due to the CU events}, 
can refer to any people that are injured or died in the civil unrest events, including 
{\em both agents and other types of people}. If you are not certain based on the text descriptions 
that some people were injured or died, do not annotate them. 
For example, if you see 
  \begin{quote}
   {\em Two guards was hit by stones.}
  \end{quote}
in the context of the CU events, 
 ``Two guards'' should be annotated because people generally get hurt when hit by stones.
 However, given 
   \begin{quote}
    {\em A policeman was hit by eggs.}
   \end{quote}
in the context of the CU events, 
``A policeman'' should not be annotated because people generally would not be hurt when hit by eggs.

\item[{\bf (7)}] {\bf PhysicalEffects (Damages or Destructions)} 

Buildings or property that are physically 
damaged or destroyed {\em due to the CU events}. 
For instance, given 
\begin{quote}
 {\em They stormed an airport, damaged the VIP lounge and surged 
 onto the runway to prevent a flight taking off.}
\end{quote}
in the context of the CU events, 
 you should label ``the VIP lounge'' as one filler. 

\item[{\bf (8)}] {\bf Instruments/Weapons} 

Anything that is used by {\em both agents and other types of people} 
{em during the CU events} with the intent to injure others, damage property, control the 
crowd or defend themselves. Weapons can be stones, bombs batons, and teargas. 
For instance, if you see 
\begin{quote}
 {\em Police used tear gas, fire hoses and pepper spray to 
 hold back hundreds of demonstrators led by militant Korean farmers, some of whom were 
 armed with bamboo sticks and metal bars .}
\end{quote}
 in the context of the CU events, 
 you should label ``tear gas'', ``fire hoses'', ``pepper spray'', ``bamboo sticks'' 
 and ``metal bars'' as the weapons. 
 \end{enumerate}

{\bf NOTES}
\begin{enumerate}
\item[{\bf 1.}] Please label appropriate strings in both headlines and body texts, 
but with preference to strings from body texts. 
If string A from the headline and string B from the body text are equally informative 
to be a role filler, please label string B instead of A. If an appropriate role filler C
is only seen in the headline, please label C. 
\item[{\bf 2.}] For slots Agent/Population, Site/Facility, HumanEffects, PhysicalEffects and 
Instruments/Weapons, annotations should be complete base noun phrases that include 
head nouns, modifiers, determiners and articles. 
\end{enumerate}