% Complete in less than 350 words.

%This file is used to maintain the abstract.  Until something is properly written here,
%the abstract shall be undergoing initial phases of contemplation. Please be patient as adequate
%time is required to allow this section to properly bloom into its full existance.

Events are one important type of information throughout the text. 
%Accurately extracting significant events from large volumes of text 
%informs the government, companies and the public regarding 
%possible changing circumstances caused or implied by events. 
Event extraction 
%in natural language processing 
is an information extraction (IE)
task that involves identifying 
%noun phrases%
entities and objects (mainly noun phrases) that represent 
%different key roles for
important roles in
events 
%in a particular domain.
of a particular type.
However, the extraction performance of current event extraction systems 
is limited because they mainly consider local context 
%(mostly from the same sentence as the extracted noun phrase) 
(mostly isolated sentences) when making extraction decisions. 

%because current event extraction systems mainly consider local context 
%(mostly from the same sentence as the extracted noun phrase) 
%(mostly isolated sentences) when making extraction decisions, 
%and ignore the influences of wider contexts. 
%their performance is limited on two aspects. 
%First, the coverage of event extraction systems is 
%limited because many role fillers occur in contexts
%that do not explicitly mention the event. 
%Second, the accuracy of current event extraction systems is 
%also limited because 
%many local context 
%explicitly refer to a relevant event.
%contains 
%seemingly relevant event keywords or phrases may not 
%be 
%truly about the 
%referring to a relevant event due to ambiguity and metaphor. 


%from the discourse. 
My research first aims to improve both coverage and accuracy of event extraction performance 
%by exploring discourse guided models.
by incorporating discourse information 
beyond an individual sentence 
%to effectively identify event contexts.  
%Extracting event information completely and accurately 
%is challenging mainly due to the high complexity of 
%discourse phenomena. 
%In this dissertation, I present 
%two discourse-guided event extraction architectures 
%that explore evidence and clues from wider discourse 
to seek out or validate pieces of event descriptions. 
%Specifically, 
{\it TIER} is a multilayered event extraction architecture 
that performs text analysis at multiple granularities 
to progressively "zoom in" on relevant event information. 
%By 
%{\it TIER} represents a two-pronged strategy for event extraction that 
%distinguish two types of documents that mention the relevant events 
%{\it event narratives}  v.s {\it fleeting references}, 
%and only consider to extract information from secondary contexts in 
%{\it event narratives}. 
%incorporates both document genre
%and role-specific context recognition 
%to extract event formation from a variety of different contexts.
{\it LINKER} is a unified discourse-guided approach 
%that models textual cohesion properties in a single 
%structured sentence classifier. 
that includes a structured sentence classifier 
to sequentially read a
story and determine which sentences contain event
information based on both the local and preceding contexts. 
%The structured sentence classifier 
%uses well designed features to 
%can model textual cohesion properties across sentences, 
%including lexical word
%associations, discourse relations across sentences and 
%distributional properties of the candidate role fillers within
%and across sentences. 
Experimental results on two distinct event domains 
show that compared to previous 
event extraction systems, {\it TIER} 
can recover more event information, while maintaining a good extraction precision.
and {\it LINKER} 
can further improve extraction accuracy.  
%and the overall extraction performance. 
%significantly outperformed 
%the previous event extraction systems and my first 
%discourse-guided event extraction model {\it TIER}.

Another issue of current event extraction systems is 
that they do not distinguish if the documents being 
analyzed contain any relevant information or not. 
Processing documents that do not mention a relevant event 
%for extraction purpose 
is 
%meaningless and 
a waste of computing resources and 
%any extractions from event irrelevant documents 
%that do not contain a relevant event 
%will be false and hurt event extraction accuracy.
can affect extraction accuracy negatively 
%Furthermore, 
%accurate event recognition will improve event extraction accuracy 
because any extractions from event irrelevant documents
will be false. 
However, finding documents that describe a specific type of event 
is also challenging because of the wide variety and 
ambiguity of event expressions. 
In this dissertation, I present 
the multi-faceted event recognition approach that uses 
event defining characteristics (facets), 
in addition to event expressions, to effectively 
resolve the complexity of event descriptions. I also 
present a novel bootstrapping algorithm to automatically 
learn event expressions as well as facets of events, 
which requires minimal human supervision. 
Experimental results show that the multi-faceted event recognition approach 
can effectively identify documents that describe a particular type of event and 
make event extraction systems more precise. 
%from unannotated texts. 
%which will enable fast configurations of domain-specific event detection systems.