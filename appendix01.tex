\chapter{CU Event Document Annotation Guidelines}
\label{Appendix:EventDocumentAnnotationGuidelines}
This appendix lists the civil unrest event document annotation guidelines that were provided
to the annotators.

\section{Annotation Task Definition}

You will need to read a set of news articles and determine which articles discuss a CU event.
If an article mentions at least one CU event, label it as a CU\_article. 
If an article mentions no CU event, label it as an Irrel\_article.\\

\section{Civil Unrest Event Definition}

Civil unrest (CU) is a broad term that is typically used by the media or
law enforcement to describe a form of public disturbance caused by a
group of people for a purpose. Civil unrest events include activities to protest
against major socio-political problems, events of activism to support
a cause (e.g., peace rallies or large-scale marches to support a
prominent figure), and events to promote changes in government or
business affairs (e.g., large gatherings to rally for higher wages).
Types of civil unrest can include, but are not necessarily limited to:
strikes, rallies, sit-ins and other forms of obstructions, riots,
sabotage, and other forms of public disturbance motivated by a
cause. It is intended to be a demonstration to the public, the
government, or an institution (e.g., business or educational sectors),
but can sometimes escalate into general chaos.\\

{\bf Civil Unrest Events to be Annotated}
%CU Events to be Considered:
\begin{enumerate}
\item[{\bf 1.}]According to the definition, CU events do not include war, ethnic fightings 
or fightings involving armed parties only. 
\item[{\bf 2.}] CU events include mentions of currently on-going and 
recent (within one year) CU events.
Old CU events that happened more than one year ago from the date when the article was 
published should not be labeled.
\item[{\bf 3.}] CU events include mentions of threatened and planned CU events that 
may happen even if there is uncertainty or they are mentioned conditionally. 
CU events do not include mentions of threatened and planned CU events that
will definitely not happen.
\item[{\bf 4.}] CU events do not include purely hypothetical or abstract mentions 
of CU events or activities. 
These events are mentioned in general discussions or metaphorically.
\item[{\bf 5.}] CU events can be described only in a small portion of a text 
and may not be the focus of the text.
However, if an article only mentions CU events in a single noun phrase fleetingly, 
e.g., ``last year's teachers strike'', ``a possible student protest'', 
no any other detail about those events are mentioned,
they are treated as abstract mentions of CU events and the article is an Irerel\_article.
\item[{\bf 6.}] Event summary information that is synthesized from two or more events 
should NOT be annotated.

\end{enumerate}





